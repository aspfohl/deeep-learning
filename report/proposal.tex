\documentclass{article}

% if you need to pass options to natbib, use, e.g.:
%     \PassOptionsToPackage{numbers, compress}{natbib}
% before loading neurips_2018

% ready for submission
% \usepackage{neurips_2018}

% to compile a preprint version, e.g., for submission to arXiv, add add the
% [preprint] option:
\usepackage[preprint]{neurips_2018}

% to compile a camera-ready version, add the [final] option, e.g.:
\usepackage[final]{neurips_2018}

% to avoid loading the natbib package, add option nonatbib:
%     \usepackage[nonatbib]{neurips_2018}

\usepackage[utf8]{inputenc} % allow utf-8 input
\usepackage[T1]{fontenc}    % use 8-bit T1 fonts
\usepackage{hyperref}       % hyperlinks
\usepackage{url}            % simple URL typesetting
\usepackage{booktabs}       % professional-quality tables
\usepackage{amsfonts}       % blackboard math symbols
\usepackage{nicefrac}       % compact symbols for 1/2, etc.
\usepackage{microtype}      % microtypography

\title{Title}

% The \author macro works with any number of authors. There are two commands
% used to separate the names and addresses of multiple authors: \And and \AND.
%
% Using \And between authors leaves it to LaTeX to determine where to break the
% lines. Using \AND forces a line break at that point. So, if LaTeX puts 3 of 4
% authors names on the first line, and the last on the second line, try using
% \AND instead of \And before the third author name.

\author{%
  Mahir Kothary\\
  Computer Science\\
  Cornell Tech\\
  NYC, NY 10044 \\
  \texttt{mk942@cornell.edu} 
 \And Anna Pfohl\\
  Computer Science\\
  Cornell Tech\\
  NYC, NY 10044 \\
  \texttt{als462@cornell.edu}
\AND
  Gustavo Vasquez \\
  Computer Science\\
  Cornell Tech\\
  NYC, NY 10044 \\
  \texttt{av384@cornell.edu}
\And
  Jia Zhao \\
  Computer Science\\
  Cornell Tech\\
  NYC, NY 10044 \\
  \texttt{jz538@cornell.edu}
}

\begin{document}
% \nipsfinalcopy is no longer used

\maketitle

\begin{abstract}
  The abstract paragraph should be indented \nicefrac{1}{2}~inch (3~picas) on
  both the left- and right-hand margins. Use 10~point type, with a vertical
  spacing (leading) of 11~points.  The word \textbf{Abstract} must be centered,
  bold, and in point size 12. Two line spaces precede the abstract. The abstract
  must be limited to one paragraph.
\end{abstract}

\section{Introduction}

What is the problem you are working on and why is it important? 

\subsection{Related Work}

What’s been done on the problem so far? What methods have been tried? Can you list papers that have done  similar things and talk about that?

Definitely make it clear that you have looked at papers doing similar things more words in the paper compared to earlier years.

\section{Proposed Work}

What are you going to do? Are you going to compare methods? Are you going to change methods? What methods  will you be comparing to? What metrics will you use to know if you succeed?

\subsection{Datasets}
What datasets are you going to use?

\subsection{Evaluation}
What results did others get on those datasets?

\subsection{Timeline}


% \begin{figure}
%   \centering
%   \fbox{\rule[-.5cm]{0cm}{4cm} \rule[-.5cm]{4cm}{0cm}}
%   \caption{Sample figure caption.}
% \end{figure}

% \begin{table}
%   \caption{Sample table title}
%   \label{sample-table}
%   \centering
%   \begin{tabular}{lll}
%     \toprule
%     \multicolumn{2}{c}{Part}                   \\
%     \cmidrule(r){1-2}
%     Name     & Description     & Size ($\mu$m) \\
%     \midrule
%     Dendrite & Input terminal  & $\sim$100     \\
%     Axon     & Output terminal & $\sim$10      \\
%     Soma     & Cell body       & up to $10^6$  \\
%     \bottomrule
%   \end{tabular}
% \end{table}


\end{document}
